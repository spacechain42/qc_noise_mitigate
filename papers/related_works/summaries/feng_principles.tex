% Likely just used as reference for intro/background section
\subsection{Quantum Computing Principles}
\label{sec:fengPrinciples}
The base unit of information in a quantum computer is a qubit, analogous to a bit in classical computing.
A qubit, once measured has possible values of $\dirac{0}$ or $\dirac{1}$.
Before measurement a qubit exists in a superposition of all possible states with some probability of collapsing into each state once observed.
For a two-qubit system, the states are all possible combinations of the individual qubit states $\dirac{00}, \dirac{01}, \dirac{10}, \dirac{11}$ where the probabilities of each state summing to 1.
\cite{feng_quantum_2023}
