\subsection{System Noise Fingerprinting}
\label{sec:martinaFingerprint}
Martina et al demonstrate that the overall noise fingerprint of a quantum computer can be used to distinguish it from other devices.
The noise fingerprint is a time-ordered sequence of output probabilities at nine points in a specified quantum circuit.
Repeating these circuit measurements either immediately sequential or with delay provides a characterization of how the noise fingerprint of a device evolves over time.
This noise fingerprint is used as the input for a support vector machine (SVM) classifier with the output label of which device was used for the calculation.
Using the trained SVM model, the seven IBM devices used for training could be distinguished with over 99\% accuracy after only three measurements.
This ease of device differentiation based on noise fingerprint indicates that any noise modeling must be both device and time specific - any noise mitigation model for a device must be re-trained several times each day to maintain accuracy.
\cite{martina_learning_2022}
